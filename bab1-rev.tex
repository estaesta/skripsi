%%%%%%%%%%%%%%%%%%%%%%%%%%%%%%%%%%%%%%%%%%%%%%%%%%%%%%%%%%%%%%%%%%%%%%%
% BAB 1
%%%%%%%%%%%%%%%%%%%%%%%%%%%%%%%%%%%%%%%%%%%%%%%%%%%%%%%%%%%%%%%%%%%%%%%
\mychapter{1}{BAB 1 PENDAHULUAN}

\section{Latar Belakang Masalah}

Penyakit kardiovaskular (CVD), juga dikenal sebagai penyakit kardiovaskular, merupakan penyebab utama kematian di seluruh dunia. Pada tahun 2019, diperkirakan terdapat 17,9 juta kematian akibat CVD \parencite{worldhealthorganizationCardiovascularDiseasesCVDs2021}. Angka ini mencakup sekitar 32\% dari total kematian global. Dalam menghadapi angka kematian yang tinggi ini, tindakan pencegahan harus diambil untuk mengurangi risiko kematian akibat CVD.

Pemantauan detak jantung adalah salah satu aspek penting dalam diagnosis dan pemantauan kondisi kesehatan.
Informasi yang diperoleh dari pemantauan detak jantung dapat digunakan untuk mendeteksi gangguan pada jantung.
Deteksi dini gangguan jantung sangat penting untuk mengambil tindakan pencegahan atau pengobatan yang tepat, dengan tujuan mengurangi risiko kematian akibat penyakit kardiovaskular.

Long Short-Term Memory (LSTM) adalah arsitektur jaringan saraf tiruan yang dikembangkan dari Recurrent Neural Network (RNN). LSTM dikenal karena kemampuannya dalam mengelola data sekuensial atau berurutan. Kemampuan ini menjadikan LSTM sangat cocok untuk memantau dan memprediksi detak jantung, yang merupakan data sekuensial.

Inferensi adalah proses penggunaan model yang telah dilatih untuk melakukan prediksi pada data baru. Efisiensi inferensi sangat penting agar model dapat digunakan secara luas pada berbagai perangkat \parencite{ulkerReviewingInferencePerformance2020}. Pengujian performa inferensi pada berbagai perangkat dapat memberikan informasi tentang efisiensi model tersebut pada perangkat yang berbeda.

Penelitian sebelumnya telah menunjukkan bahwa model LSTM dapat berhasil digunakan untuk memprediksi detak jantung dari data elektrokardiogram (ECG), yang merupakan data sekuensial \parencite{shchetininArrhythmiaDetectionUsing2022}. Namun, implementasi dan pengujian performa pada perangkat nyata dapat memberikan informasi yang lebih akurat tentang performa model LSTM dalam memprediksi detak jantung dalam kasus dunia nyata.

Pada penelitian lain, \textcite{ahsanuzzamanLowCostPortable2020} mengusulkan sistem alarm dan pemantauan Elektrokardiogram (ECG) pada perangkat portabel. Sistem ini fokus pada prediksi aritmia (fibrilasi atrium) dengan menggunakan arsitektur LSTM yang diintegrasikan pada Raspberry Pi 3. Sistem ini menggunakan Arduino Uno sebagai perantara antara sensor dan Raspberry Pi, dan data ECG ditampilkan pada perangkat Android.

Berdasarkan latar belakang di atas, penelitian ini bertujuan untuk mengembangkan dan mengimplementasikan model prediksi detak jantung berbasis LSTM. Selain itu, penelitian ini juga akan mengevaluasi performa model prediksi detak jantung berbasis LSTM pada berbagai perangkat, dengan memperhatikan akurasi, presisi, recall, F1-score, waktu inferensi, dan penggunaan memori.

\section{Rumusan Masalah}

Berdasarkan latar belakang di atas, rumusan masalah penelitian ini adalah sebagai berikut:

\begin{enumerate}
  \item Bagaimana nilai akurasi, presisi, recall, dan F1-score dari model prediksi detak jantung berbasis LSTM?
  \item Bagaimana waktu inferensi dan penggunaan memori dari model prediksi detak jantung berbasis LSTM pada berbagai perangkat?
\end{enumerate}

\section{Tujuan Penelitian}

Tujuan dari penelitian ini adalah sebagai berikut:

\begin{enumerate}
  \item Mengetahui nilai akurasi, presisi, recall, dan F1-score dari model prediksi detak jantung berbasis LSTM.
  \item Mengetahui waktu inferensi dan penggunaan memori dari model prediksi detak jantungberbasis LSTM pada berbagai perangkat.
\end{enumerate}

\section{Manfaat Penelitian}

Manfaat dari penelitian ini adalah sebagai berikut:

\begin{enumerate}
  \item Memberikan kontribusi dalam pengembangan metode prediksi detak jantung berbasis LSTM.
  \item Memberikan informasi tentang performa model prediksi detak jantung berbasis LSTM pada berbagai perangkat.
  \item Menyediakan dasar untuk pengembangan sistem pemantauan detak jantung yang efisien dan akurat.
\end{enumerate}

\section{Batasan Penelitian}

Penelitian ini memiliki batasan sebagai berikut:

\begin{enumerate}
  \item Penelitian ini hanya menggunakan data detak jantung dalam bentuk elektrokardiogram (ECG).
  \item Pengujian performa model dilakukan pada perangkat Raspberry Pi 3, Raspberry Pi 4, dan komputer dengan spesifikasi tertentu.
  \item Penelitian ini tidak melibatkan penggunaan data detak jantung dalam bentuk lain, seperti photoplethysmogram (PPG).
\end{enumerate}

\section{Sistematika Penulisan}

Penulisan laporan penelitian ini terdiri dari lima bab, yaitu:

\begin{enumerate}
  \item Bab 1: Pendahuluan \\
        Bab ini berisi latar belakang masalah, rumusan masalah, tujuan penelitian, manfaat penelitian, batasan penelitian, dan sistematika penulisan.
  \item Bab 2: Tinjauan Pustaka \\
        Bab ini berisi tinjauan pustaka tentang detak jantung, penyakit kardiovaskular, pemantauan detak jantung, arsitektur LSTM, dan pengujian performa inferensi pada berbagai perangkat.
  \item Bab 3: Metode Penelitian \\
        Bab ini berisi penjelasan tentang metode yang digunakan dalam penelitian ini, seperti pengumpulan data, pemrosesan data, pengembangan model, dan pengujian performa.
  \item Bab 4: Hasil dan Analisis \\
        Bab ini berisi hasil dari penelitian yang dilakukan, serta analisis dan interpretasi hasil tersebut.
  \item Bab 5: Kesimpulan dan Saran \\
        Bab ini berisi kesimpulan dari penelitian yang dilakukan, serta saran untuk penelitian selanjutnya.
\end{enumerate}
