%%%%%%%%%%%%%%%%%%%%%%%%%%%%%%%%%%%%%%%%%%%%%%%%%%%%%%%%%%%%%%%%%%%%%%%
% BAB 1
%%%%%%%%%%%%%%%%%%%%%%%%%%%%%%%%%%%%%%%%%%%%%%%%%%%%%%%%%%%%%%%%%%%%%%%
\mychapter{1}{BAB 1 PENDAHULUAN}

\section{Latar Belakang Masalah}

\emph{Cardiovaskular disease} (CVD) atau penyakit kardiovaskular merupakan penyebab utama kematian di dunia.
Pada tahun 2019, diperkirakan terdapat 17,9 juta kematian yang diakibatkan oleh CVD \parencite{worldhealthorganizationCardiovascularDiseasesCVDs2021}.
Angka kematian tersebut mencakup sekitar 32\% dari total kematian di dunia. 
Dengan angka kematian yang begitu tinggi, diperlukan langkah preventif untuk dapat mencegah terjadinya kematian akibat CVD.
Salah satu langkah preventif yang dapat dilakukan adalah dengan melakukan pemantauan detak jantung.

% Kesehatan merupakan elemen yang sangat penting dalam kehidupan
% manusia. Kesadaran akan kebutuhan untuk memantau kondisi kesehatan
% pribadi telah menjadi fokus utama bagi banyak orang. Teknologi pemantauan
% kesehatan portabel muncul sebagai solusi untuk menyediakan akses yang mudah
% dan terjangkau terhadap data kesehatan individu. 

Pemantauan detak jantung adalah salah satu aspek penting dalam diagnosis dan pemantauan kondisi kesehatan.
% Pemantauan detak jantung memberikan informasi vital yang dapat digunakan untuk mendeteksi gangguan pada jantung.
Informasi yang diperoleh dari pemantauan detak jantung dapat digunakan untuk mendeteksi gangguan pada jantung.
Deteksi dini gangguan jantung sangat penting untuk mengambil tindakan pencegahan atau pengobatan yang tepat, sehingga dapat mengurangi risiko kematian akibat penyakit kardiovaskular.
 % Structure of the heart and its electrical functionality can be analyzed from the electrocardiogram (ECG), which is treated as a golden standard for clinical diagnosis (such as: analysis of heartbeats, biometric identification, and emotion recognition etc.)
Metode yang umum digunakan untuk memantau detak jantung adalah dengan menggunakan sensor \emph{electrocardiogram} (ECG).

% --- penjelasan tentang ECG, lstm beserta penelitian terdahulu, inferencing, dan penelitian terdahulu
% ecg ada blablabla
% ecg dapat mendeteksi cvd
% kelemahan ecg
% berkembang analisis data ecg otomatis
% penelitian terdahulu berhasil menggunnakan lstm
% lstm adalah blabla dan kelebihan lstm
% -
% inferencing
% for that reason, 

\textit{Electrocardiogram} (ECG) atau juga dikenal dengan elektrokardiogram (EKG) adalah rekaman listrik dari aktivitas jantung yang direkam secara non-invasif melalui elektroda yang ditempatkan pada kulit \parencite{sattarElectrocardiogram2024}.
Rekaman ini berbentuk sinyal yang merepresentasikan aktivitas listrik jantung selama periode waktu tertentu.
Sinyal ECG memiliki pola tertentu yang dapat digunakan untuk mendiagnosis berbagai gangguan jantung.
% Analisis data ECG umumnya dilakukan oleh dokter atau tenaga medis yang berpengalaman.
Untuk mendiagnosis gangguan jantung, analisis data ECG umumnya dilakukan oleh dokter atau tenaga medis yang berpengalaman.
Akan tetapi, analisis ECG secara manual memerlukan waktu yang lama, melelahkan, dan membutuhkan biaya yang tinggi \parencite{anbalaganAnalysisVariousTechniques2023}.
Untuk mengatasi permasalahan tersebut, penelitian mengenai analisis data ECG secara otomatis telah berkembang pesat.

% \textit{Long Short-Term Memory} (LSTM) merupakan arsitektur jaringan saraf tiruan pengembangan dari \emph{Recurrent Neural Network} (RNN).
% LSTM terkenal akan kemampuannya dalam mengelola data sekuensial atau berurutan.
% Hal ini membuat LSTM sangat sesuai untuk memantau dan memprediksi detak jantung yang merupakan data sekuensial.
% Penelitian sebelumnya telah menunjukkan bahwa model LSTM dapat digunakan dengan sukses untuk memprediksi detak jantung dari data elektrokardiogram (ECG) yang merupakan data sekuensial \parencite{shchetininArrhythmiaDetectionUsing2022}. Implementasi dan pengujian performa pada perangkat nyata dapat memberikan informasi yang lebih akurat tentang performa model LSTM dalam memprediksi detak jantung pada kasus nyata.

% \textcite{shchetininArrhythmiaDetectionUsing2022} telah mengembangkan model prediksi detak jantung berbasis LSTM yang mampu memprediksi detak jantung dari data ECG.
\textcite{shchetininArrhythmiaDetectionUsing2022} telah membuktikan bahwa model \textit{deep learning} berbasis \emph{Long Short-Term Memory} (LSTM) dapat digunakan untuk melakukan klasifikasi detak jantung menggunakan data ECG.
Model tersebut mampu mengklasifikasikan detak jantung ke dalam lima kelas dengan akurasi yang tinggi.
LSTM merupakan arsitektur jaringan saraf tiruan yang dikembangkan dari \emph{Recurrent Neural Network} (RNN).
LSTM terkenal akan kemampuannya dalam mengelola data sekuensial atau berurutan.
Hal ini membuat LSTM cocok digunakan untuk memprediksi detak jantung yang merupakan data sekuensial.

% edge device?
% Penelitian tentang implementasi model prediksi detak jantung pada perangkat tepi juga mulai banyak dilakukan.
% Perangkat tepi (\emph{edge device}) merupakan perangkat yang memiliki kemampuan komputasi dan penyimpanan data yang terbatas.
Salah satu tantangan dalam implementasi model prediksi detak jantung adalah efisiensi dalam melakukan inferensi.
Inferensi merupakan proses penggunaan model yang telah dilatih untuk memprediksi data baru. 
Efisiensi inferensi sangat penting untuk memungkinkan penggunaan model pada perangkat yang luas \parencite{ulkerReviewingInferencePerformance2020}. 
Pengujian performa inferensi pada beberapa perangkat dapat memberikan informasi mengenai efisiensi model tersebut pada perangkat yang berbeda-beda.
Model dengan performa inferensi yang baik dapat digunakan pada perangkat yang memiliki keterbatasan daya komputasi serta memberikan pengalaman pengguna yang lebih baik karena waktu inferensi yang lebih cepat.
Oleh karena itu, penting untuk melakukan evaluasi performa model prediksi detak jantung berbasis LSTM pada beberapa perangkat.

Berdasarkan latar belakang di atas, pada penelitian ini akan dikembangkan dan diimplementasikan model prediksi detak jantung berbasis LSTM pada beberapa perangkat.
Penelitian ini juga akan melakukan evaluasi performa model prediksi detak jantung berbasis LSTM pada beberapa perangkat baik dari segi akurasi, presisi, \emph{recall}, dan \emph{F1-score} maupun dari segi efisiensi inferensi model pada perangkat yang berbeda-beda.
% Pada penelitian ini juga akan dilakukan evaluasi performa model prediksi detak jantung berbasis LSTM pada beberapa perangkat untuk mengetahui baik akurasi model maupun efisiensi inferensi model pada perangkat yang berbeda-beda.


% Pada penelitian lain, \textcite{ahsanuzzamanLowCostPortable2020} mengusulkan suatu sistem alarm dan pemantauan Elektrokardiogram (ECG) pada perangkat portabel. Fokus utama sistem tersebut adalah pada prediksi aritmia (fibrilasi atrium) dengan mengadopsi arsitektur LSTM yang diintegrasikan pada Raspberry Pi 3. Sebagai perantara antara sensor dengan Raspberry Pi, sistem tersebut menggunakan Arduino Uno. Perangkat Android juga digunakan untuk menampilkan data ECG. 

% Berdasarkan latar belakang di atas, penelitian ini akan mengembangkan dan mengimplementasikan model prediksi detak jantung berbasis LSTM.
% Selain itu, penelitian ini juga akan melakukan evaluasi model prediksi detak jantung berbasis LSTM pada beberapa perangkat untuk mengetahui performa model tersebut baik
% akurasi, presisi, \emph{recall}, dan \emph{F1-score} maupun waktu inferensi serta penggunaan memori.


\section{Rumusan Masalah}

Berdasarkan latar belakang di atas maka dilakukan penyusunan rumusan masalah sebagai berikut:

\begin{enumerate}
  % \item Bagaimana mengembangkan dan mengimplementasikan model prediksi detak jantung berbasis LSTM pada Raspberry Pi.
  % \item Bagaimana hasil evaluasi model prediksi detak jantung berbasis LSTM pada Raspberry Pi.
  \item Bagaimana nilai akurasi, presisi, \emph{recall}, dan \emph{F1-score} dari model prediksi detak jantung berbasis LSTM.
  \item Bagaimana waktu inferensi serta penggunaan memori dari model prediksi detak jantung berbasis LSTM pada beberapa perangkat.
  % \item Bagaimana efisiensi inferensi model prediksi detak jantung berbasis LSTM pada beberapa perangkat.
\end{enumerate}


\section{Tujuan Penelitian}
Tujuan dilakukannya penelitian ini adalah sebagai berikut:

\begin{enumerate}
  % \item Mengembangkan dan mengimplementasikan model prediksi detak jantung berbasis LSTM pada Raspberry Pi.
  % \item Melakukan evaluasi model prediksi detak jantung berbasis LSTM pada Raspberry Pi.
  \item Mengetahui nilai akurasi, presisi, \emph{recall}, dan \emph{F1-score} dari model prediksi detak jantung berbasis LSTM.
  \item Mengetahui waktu inferensi serta penggunaan memori dari model prediksi detak jantung berbasis LSTM pada beberapa perangkat.
  % \item Mengetahui efisiensi inferensi model prediksi detak jantung berbasis LSTM pada beberapa perangkat.
\end{enumerate}


\section{Manfaat Penelitian}

% Manfaat dari penelitian ini adalah mengetahui hasil dari pengembangan dan implementasi model prediksi detak jantung berbasis LSTM, serta mengetahui hasil pengujian dan evaluasi model prediksi detak jantung berbasis LSTM pada beberapa perangkat. Hasil penelitian ini diharapkan dapat memberikan informasi yang berguna dalam pengembangan sistem pemantauan detak jantung berbasis LSTM.
Manfaat dari penelitian ini adalah mengetahui performa model prediksi detak jantung berbasis LSTM pada beberapa perangkat, baik dari segi akurasi, presisi, \emph{recall}, dan \emph{F1-score} maupun efisiensi inferensi model pada perangkat yang berbeda-beda. Hasil penelitian ini diharapkan dapat memberikan informasi yang berguna dalam pengembangan sistem pemantauan detak jantung.


\section{Batasan Masalah}

Pada penelitian ini terdapat batasan masalah yang bertujuan untuk memfokuskan penelitian ini. Batasan masalah penelitian ini adalah sebagai berikut:
\begin{enumerate}
  \item Prediksi yang dilakukan pada penelitian terbatas pada lima kelas sesuai rekomendasi AAMI (Association for the Advancement of Medical Instrumentation).
  \item Pengujian yang dilakukan menggunakan data detak jantung yang telah diperoleh sebelumnya menggunakan sensor ECG Polar H10.
  \item Deteksi detak jantung dilakukan secara \textit{batch}.
    % , bukan secara \textit{real-time}.
\end{enumerate}




\section{Sistematika Pembahasan}
\noindent
\textbf{BAB I : PENDAHULUAN}

Pada bab ini dijelaskan latar belakang, rumusan masalah, batasan,
tujuan, manfaat,  dan sistematika penulisan.\\

\noindent
% \textbf{BAB II : TINJAUAN PUSTAKA DAN LANDASAN TEORI}
\textbf{BAB II : LANDASAN KEPUSTAKAAN}

Pada bab ini dijelaskan teori-teori dan penelitian terdahulu yang
digunakan sebagai acuan dan dasar dalam penelitian.\\

\noindent
\textbf{BAB III : METODOLOGI PENELITIAN}

Pada bab ini dijelaskan metode yang digunakan dalam penelitian
meliputi langkah kerja, pertanyaan penelitian, alat dan bahan, serta
tahapan dan alur penelitian.\\

% \noindent
% \textbf{BAB IV : HASIL DAN PEMBAHASAN}
%
% Pada bab ini dijelaskan hasil penelitian dan pembahasannya.\\

\noindent
\textbf{BAB IV : PEMBUATAN MODEL}

Pada bab ini dijelaskan tentang pembuatan model prediksi detak jantung berbasis LSTM.\\

\noindent
\textbf{BAB V : IMPLEMENTASI DAN PENGUJIAN}

Pada bab ini dijelaskan tentang implementasi model prediksi detak jantung berbasis LSTM pada beberapa perangkat dan pengujian performa inferensi model.\\

\noindent
\textbf{BAB VI : KESIMPULAN DAN SARAN}

Pada bab ini ditulis kesimpulan akhir dari penelitian dan saran untuk
pengembangan penelitian selanjutnya.\\
