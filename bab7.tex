%%%%%%%%%%%%%%%%%%%%%%%%%%%%%%%%%%%%%%%%%%%%%%%%%%%%%%%%%%%%%%%%%%%%%%%
% BAB 7
%%%%%%%%%%%%%%%%%%%%%%%%%%%%%%%%%%%%%%%%%%%%%%%%%%%%%%%%%%%%%%%%%%%%%%%





\mychapter{7}{BAB 7 PENUTUP}

\section{Kesimpulan}
Berdasarkan hasil penelitian tentang evaluasi performa model klasifikasi detak jantung berbasis LSTM pada perangkat tepi yang telah dilakukan, dapat diambil kesimpulan sebagai berikut:
% make the numbering no indent
\begin{enumerate}[leftmargin=0.5cm]
  \item Model klasifikasi detak jantung berbasis LSTM yang dikembangkan mampu mengklasifikasikan detak jantung dengan baik dan akurat, dengan nilai akurasi, presisi, \emph{recall}, dan \emph{F1-score} tertinggi masing-masing sebesar 0,9679, 0,9662, 0,9679, dan 0,9652, yang diperoleh oleh model LSTM-512.
  % tentang evaluasi performa inferensi
  \item Evaluasi performa inferensi model klasifikasi detak jantung berbasis LSTM pada perangkat tepi menunjukkan hasil yang baik dengan uraian sebagai berikut:
    % TODO: perjelas. jangan hanya angka, berikan penjelasan
    % model kecepatan baik, seluruh di bawah 20ms
    % model memori baik, seluruh di bawah 200MB
  \begin{enumerate}
    \item Model LSTM-512, LSTM-256, BiLSTM, dan LSTM-FCN mampu menghasilkan waktu inferensi yang baik, dengan waktu inferensi di bawah 20 ms per detak jantung.
    % \item Model LSTM-FCN menghasilkan waktu inferensi tercepat senilai 12.0372 ms per detak jantung pada perangkat tepi Raspberry Pi 4 Model B.
% LSTM-512       & 15.8634                   \\
% LSTM-256       & 13.6235                   \\
% BiLSTM         & 14.3540                   \\
% LSTM-FCN       & \textbf{12.0372}                   \\ \hline
    % \item Pada perangkat tepi Raspberry Pi 4 Model B, model LSTM-FCN menghasilkan waktu inferensi tercepat senilai 12.0372 ms per detak jantung
      % , diikuti oleh model LSTM-256 dengan waktu inferensi 13.6235 ms per detak jantung, model BiLSTM dengan waktu inferensi 14.3540 ms per detak jantung, dan model LSTM-512 dengan waktu inferensi 15.8634 ms per detak jantung.
% \textbf{Model} & \textbf{Waktu Inferensi per Detak Jantung (ms)} \\ \hline
% LSTM-512       & 3.9189                   \\
% LSTM-256       & 2.9050                   \\
% BiLSTM         & 3.3193                   \\
% LSTM-FCN       & \textbf{2.8435}                   \\ \hline
    % \item Pada perangkat tepi Intel NUC, model LSTM-FNC menghasilkan waktu inferensi tercepat senilai 2.8435 ms per detak jantung
    \item Model LSTM-FCN menghasilkan waktu inferensi tercepat baik pada perangkat tepi Raspberry Pi 4 Model B maupun Intel NUC, dengan waktu inferensi masing-masing sebesar 12,0372 ms dan 2,8435 ms per detak jantung.
      % , diikuti oleh model LSTM-256 dengan waktu inferensi 2.9050 ms per detak jantung, model BiLSTM dengan waktu inferensi 3.3193 ms per detak jantung, dan model LSTM-512 dengan waktu inferensi 3.9189 ms per detak jantung.
    \item Model LSTM-512, LSTM-256, BiLSTM, dan LSTM-FCN memiliki penggunaan memori yang baik, dengan penggunaan memori di bawah 200MB.
    % \item Penggunaan memori antara model LSTM-512, LSTM-256, BiLSTM, dan LSTM-FCN pada perangkat tepi Raspberry Pi 4 Model B dan Intel NUC tidak menunjukkan perbedaan yang signifikan.
    \item Penggunaan memori model LSTM-512, LSTM-256, BiLSTM, dan LSTM-FCN tidak menunjukkan perbedaan yang signifikan.
    % \item Penggunaan memori pada perangkat tepi Raspberry Pi 4 Model B untuk melakukan inferensi data ECG dengan durasi 10 detik, 1 menit, dan 10 menit masing-masing sekitar 131MB, 134MB, dan 180MB.
    % \item Penggunaan memori pada perangkat tepi Intel NUC untuk melakukan inferensi data ECG dengan durasi 10 detik, 1 menit, dan 10 menit masing-masing sekitar 133MB, 136MB, dan 182MB.
    \item Penggunaan memori untuk melakukan inferensi data ECG dengan durasi 10 detik, 1 menit, dan 10 menit pada perangkat tepi Raspberry Pi 4 Model B masing-masing sekitar 131MB, 134MB, dan 180MB, sedangkan pada perangkat tepi Intel NUC masing-masing sekitar 133MB, 136MB, dan 182MB.
  \end{enumerate}
\end{enumerate}

\section{Saran}
Berdasarkan penelitian yang telah dilakukan, terdapat beberapa saran yang dapat diberikan untuk pengembangan penelitian selanjutnya, yaitu:
\begin{enumerate}[leftmargin=0.5cm]
  \item Dapat dilakukan pengujian inferensi secara \textit{real-time} untuk dapat mengetahui apakah model dapat digunakan untuk aplikasi pemantauan detak jantung secara langsung.
  % \item Dapat dilakukan optimasi model untuk dapat diimplementasikan pada perangkat tepi dengan daya komputasi yang lebih rendah.
  %perjelas optimasi -> optimasi apa? arsitektur, parameter, kuantisasi, pruning.
  % \item Dapat dilakukan optimasi model baik dengan melakukan optimasi arsitektur dan parameter model, maupun dengan melakukan kuantisasi dan \emph{pruning} model untuk dapat mencapai performa inferensi yang lebih baik.
  \item Dapat dilakukan optimasi model baik dari segi arsitektur maupun parameter model untuk dapat mencapai performa inferensi yang lebih baik.
  % \item Dapat dilakukan pengujian performa inferensi pada perangkat tepi lainnya, seperti Arduino, Jetson Nano, dan sebagainya.
  % \item Hasil prediksi detak jantung yang dihasilkan oleh model dapat dilakukan validasi lebih lanjut oleh tenaga medis.
  \item Dapat dilakukan validasi hasil prediksi detak jantung yang dihasilkan oleh model oleh tenaga medis untuk mengetahui keakuratan model pada kasus dunia nyata.
    %integrasi dengan sistem yang lebih kompleks
  % \item Dapat dilakukan integrasi model dengan sistem yang lebih kompleks, seperti sistem pemantauan detak jantung yang terhubung dengan sistem informasi rumah sakit.
\end{enumerate}
