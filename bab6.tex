%%%%%%%%%%%%%%%%%%%%%%%%%%%%%%%%%%%%%%%%%%%%%%%%%%%%%%%%%%%%%%%%%%%%%%%
% BAB 6
%%%%%%%%%%%%%%%%%%%%%%%%%%%%%%%%%%%%%%%%%%%%%%%%%%%%%%%%%%%%%%%%%%%%%%%





\mychapter{6}{BAB 6 HASIL DAN PEMBAHASAN}



% --- hasil evaluasi akurasi, presisi, recall, dan f1 score ---
Hasil evaluasi akurasi, presisi, recall, dan f1 score dari model LSTM-512, LSTM-256, BiLSTM, dan LSTM-FCN ditunjukkan pada Tabel \ref{tab:evaluasi}.
% Dari hasil evaluasi, seluruh model mampu melakukan klasifikasi dengan akurasi di atas 0.96.
Dari hasil evaluasi, model LSTM-512 memiliki akurasi, presisi, recall, dan f1 score tertinggi dibandingkan dengan model LSTM-256, BiLSTM, dan LSTM-FCN.
Model LSTM-512 memiliki akurasi sebesar 0,9679, presisi sebesar 0,9662, recall sebesar 0,9679, dan f1 score sebesar 0,9652.
Meskipun demikian, model LSTM-256, BiLSTM, dan LSTM-FCN juga mendapatkan hasil evaluasi yang baik dengan akurasi, presisi, recall, dan f1 score di atas 0,96.
Akurasi, presisi, recall, dan f1 score pada seluruh model memiliki selisih yang tidak signifikan dengan selisih maksimal sebesar 0,0046.

\begin{table}[H]
  \centering
  \caption{Hasil evaluasi akurasi, presisi, recall, dan f1 score}
  \label{tab:evaluasi}
  % \begin{tabular}{|c|c|c|c|c|}
  \begin{tabularx}{0.8\textwidth}{
      |>{\centering\arraybackslash}X
      |>{\centering\arraybackslash}X
      |>{\centering\arraybackslash}X
      |>{\centering\arraybackslash}X
      |>{\centering\arraybackslash}X|}
    \hline
    \textbf{Model} & \textbf{Akurasi} & \textbf{Presisi} & \textbf{Recall} & \textbf{F1 Score} \\ \hline
    LSTM-512       & \textbf{0,9679}           & \textbf{0,9662}          & \textbf{0,9679}         & \textbf{0,9652}           \\ 
    \hline
    LSTM-256       & 0,9674           & 0,9656          & 0,9674         & 0,9645           \\ 
    \hline
    BiLSTM         & 0,9636           & 0,9617          & 0,9636         & 0,9606           \\ 
    \hline
    LSTM-FCN       & 0,9667           & 0,9643          & 0,9667         & 0,9643           \\ \hline
  \end{tabularx}
\end{table}

% --- confussion matrix ---
Hasil evaluasi berupa \textit{confusion matrix} dari model LSTM-512, LSTM-256, BiLSTM, dan LSTM-FCN ditunjukkan pada Tabel \ref{tab:confusion-lstm512}, Tabel \ref{tab:confusion-lstm256}, Tabel \ref{tab:confusion-bilstm}, dan Tabel \ref{tab:confusion-lstmfcn}.
Dari hasil \textit{confusion matrix}, dapat dilihat bahwa model LSTM-512, LSTM-256, BiLSTM, dan LSTM-FCN memiliki kemampuan yang baik dalam melakukan klasifikasi detak jantung normal (N).
Akan tetapi, model LSTM-512, LSTM-256, BiLSTM, dan LSTM-FCN memiliki kesulitan dalam melakukan klasifikasi detak jantung jenis lain terutama detak jantung \textit{fusion} (F) dan detak jantung \textit{unknown} (Q).
Hal ini dapat dilihat dari rasio misklasifikasi pada detak jantung \textit{fusion} (F) dan detak jantung \textit{unknown} (Q) yang cukup tinggi dibandingkan dengan detak jantung lainnya.
% model cendurung mengklasifikasikan ke kelas n yang menunjukkan overfitting ke kelas n
Ketika model melakukan kesalahan klasifikasi, model cenderung mengklasifikasikan detak jantung ke kelas normal (N) yang menunjukkan adanya overfitting ke kelas normal (N).
Hal ini dapat disebabkan karena adanya ketidakseimbangan jumlah detak jantung pada masing-masing kelas pada dataset yang digunakan.
% yang menyebabkan model cenderung mengklasifikasikan detak jantung ke kelas yang mayoritas.
Ketidakseimbangan tersebut menyebabkan model cenderung mengklasifikasikan detak jantung ke kelas yang mayoritas.


\begin{table}[H]
  \centering
  \caption{Confusion matrix model LSTM-512}
  \label{tab:confusion-lstm512}
  % \begin{tabular}{cc|ccccc}
  %   \multicolumn{2}{c}{} & \multicolumn{5}{c}{Prediksi} \\
  %   & & N & S & V & F & Q \\ \hline
  %   \multirow{5}{*}{\rotatebox[origin=c]{90}{Aktual}}
  %   & N & 25722 & 25 & 166 & 7 & 0 \\
  %   & S & 88 & 641 & 79 & 0 & 0 \\
  %   & V & 335 & 36 & 1581 & 2 & 0 \\
  %   & F & 183 & 0 & 6 & 47 & 0 \\
  %   & Q & 1 & 1 & 0 & 0 & 0 \\ \hline
  % \end{tabular}
  \begin{tabularx}{0.6\textwidth}{|c
      |>{\centering\arraybackslash}X
      |>{\centering\arraybackslash}X
      |>{\centering\arraybackslash}X
      |>{\centering\arraybackslash}X
      |>{\centering\arraybackslash}X|}
    \hline
    \multirow{2}{*}{\textbf{Aktual}} & \multicolumn{5}{c|}{\textbf{Prediksi}} \\
    \cline{2-6}
               & \textbf{N} & \textbf{S} & \textbf{V} & \textbf{F} & \textbf{Q} \\ \hline
               \textbf{N} & 25722 & 25 & 166 & 7 & 0 \\
    \hline
              \textbf{S} & 88 & 641 & 79 & 0 & 0 \\
    \hline
              \textbf{V} & 335 & 36 & 1581 & 2 & 0 \\
    \hline
              \textbf{F} & 183 & 0 & 6 & 47 & 0 \\
    \hline
              \textbf{Q} & 1 & 1 & 0 & 0 & 0 \\ \hline
  \end{tabularx}
\end{table}

\begin{table}[H]
  \centering
  \caption{Confusion matrix model LSTM-256}
  \label{tab:confusion-lstm256}
  % \begin{tabular}{cc|ccccc}
  %   \multicolumn{2}{c}{} & \multicolumn{5}{c}{Prediksi} \\
  %   & & N & S & V & F & Q \\ \hline
  %   \multirow{5}{*}{\rotatebox[origin=c]{90}{Aktual}}
  %   & N & 25748 & 24 & 140 & 8 & 0 \\
  %   & S & 83 & 644 & 81 & 0 & 0 \\
  %   & V & 360 & 56 & 1538 & 0 & 0 \\
  %   & F & 182 & 0 & 8 & 46 & 0 \\
  %   & Q & 1 & 0 & 1 & 0 & 0 \\ \hline
  % \end{tabular}
  \begin{tabularx}{0.6\textwidth}{|c
      |>{\centering\arraybackslash}X
      |>{\centering\arraybackslash}X
      |>{\centering\arraybackslash}X
      |>{\centering\arraybackslash}X
      |>{\centering\arraybackslash}X|}
    \hline
    \multirow{2}{*}{\textbf{Aktual}} & \multicolumn{5}{c|}{\textbf{Prediksi}} \\
    \cline{2-6}
               & \textbf{N} & \textbf{S} & \textbf{V} & \textbf{F} & \textbf{Q} \\ \hline
               \textbf{N} & 25748 & 24 & 140 & 8 & 0 \\
    \hline
              \textbf{S} & 83 & 644 & 81 & 0 & 0 \\
    \hline
              \textbf{V} & 360 & 56 & 1538 & 0 & 0 \\
    \hline
              \textbf{F} & 182 & 0 & 8 & 46 & 0 \\
    \hline
              \textbf{Q} & 1 & 0 & 1 & 0 & 0 \\ \hline
  \end{tabularx}
\end{table}

\begin{table}[H]
  \centering
  \caption{Confusion matrix model BiLSTM}
  \label{tab:confusion-bilstm}
  % \begin{tabular}{cc|ccccc}
  %   \multicolumn{2}{c}{} & \multicolumn{5}{c}{Prediksi} \\
  %   & & N & S & V & F & Q \\ \hline
  %   \multirow{5}{*}{\rotatebox[origin=c]{90}{Aktual}}
  %   & N & 25684 & 40 & 190 & 6 & 0 \\
  %   & S & 94 & 627 & 87 & 0 & 0 \\
  %   & V & 397 & 45 & 1511 & 1 & 0 \\
  %   & F & 183 & 0 & 9 & 44 & 0 \\
  %   & Q & 1 & 0 & 1 & 0 & 0 \\ \hline
  % \end{tabular}
  \begin{tabularx}{0.6\textwidth}{|c
      |>{\centering\arraybackslash}X
      |>{\centering\arraybackslash}X
      |>{\centering\arraybackslash}X
      |>{\centering\arraybackslash}X
      |>{\centering\arraybackslash}X|}
    \hline
    \multirow{2}{*}{\textbf{Aktual}} & \multicolumn{5}{c|}{\textbf{Prediksi}} \\
    \cline{2-6}
               & \textbf{N} & \textbf{S} & \textbf{V} & \textbf{F} & \textbf{Q} \\ \hline
               \textbf{N} & 25684 & 40 & 190 & 6 & 0 \\
    \hline
               \textbf{S} & 94 & 627 & 87 & 0 & 0 \\
    \hline
               \textbf{V} & 397 & 45 & 1511 & 1 & 0 \\
    \hline
               \textbf{F} & 183 & 0 & 9 & 44 & 0 \\
    \hline
               \textbf{Q} & 1 & 0 & 1 & 0 & 0 \\ \hline
  \end{tabularx}
\end{table}

\begin{table}[H]
  \centering
  \caption{Confusion matrix model LSTM-FCN}
  \label{tab:confusion-lstmfcn}
  \begin{tabularx}{0.6\textwidth}{|c
      |>{\centering\arraybackslash}X
      |>{\centering\arraybackslash}X
      |>{\centering\arraybackslash}X
      |>{\centering\arraybackslash}X
      |>{\centering\arraybackslash}X|}
    \hline
    \multirow{2}{*}{\textbf{Aktual}} & \multicolumn{5}{c|}{\textbf{Prediksi}} \\
    \cline{2-6}
               & \textbf{N} & \textbf{S} & \textbf{V} & \textbf{F} & \textbf{Q} \\ \hline
               \textbf{N} & 25702 & 26 & 171 & 21 & 0 \\
    \hline
               \textbf{S} & 98 & 631 & 79 & 0 & 0 \\
    \hline
               \textbf{V} & 338 & 43 & 1571 & 2 & 0 \\
    \hline
               \textbf{F} & 175 & 0 & 8 & 53 & 0 \\
    \hline
               \textbf{Q} & 1 & 0 & 1 & 0 & 0 \\ \hline
  \end{tabularx}
\end{table}

% --- analisis hasil evaluasi akurasi, presisi, recall, dan f1 score ---


% --- hasil evaluasi waktu inferensi dan memory usage ---
\begin{table}[H]
\centering
\caption{Hasil evaluasi waktu inferensi dan penggunaan memori pada Raspberry Pi 4}
\label{tab:raspi4}
\begin{tabular}{|c|c|c|c|}
\hline
\textbf{Model} & \textbf{Durasi ECG} & \textbf{Waktu Inferensi (s)} & \textbf{Penggunaan Memori (MB)} \\ \hline
\multirow{3}{*}{ LSTM-512 }       & 10s                 & 0,1541                      & 131,1078                   \\ 
              \cline{2-4}
               & 1m                  & 1,0411                      & 135,0430                   \\
              \cline{2-4}
               & 10m                 & 14,0108                     & 180,4742                   \\ \hline
\multirow{3}{*}{ LSTM-256 }       & 10s                 & 0,1574                      & 131,0969                   \\ 
              \cline{2-4}
               & 1m                  & 0,9454                      & \textbf{134,2883}                   \\ 
              \cline{2-4}
               & 10m                 & 11,2050                     & 180,7773                   \\ \hline
\multirow{3}{*}{ BiLSTM }         & 10s                 & \textbf{0,1467}                      & 131,1070                   \\ 
              \cline{2-4}
               & 1m                  & 0,9615                      & 134,3352                   \\ 
              \cline{2-4}
               & 10m                 & 12,3639                     & \textbf{180,1117}                   \\ \hline
\multirow{3}{*}{ LSTM-FCN }       & 10s                 & 0,1916                      & \textbf{131,0898}                   \\
              \cline{2-4}
               & 1m                  & \textbf{0,8209}                      & 134,6477                   \\ 
              \cline{2-4}
               & 10m                 & \textbf{10,1330}                     & 180,3414                   \\ \hline
\end{tabular}
\end{table}

Tabel \ref{tab:raspi4} menunjukkan hasil evaluasi waktu inferensi dan penggunaan memori pada Raspberry Pi 4.
Pada skenario inferensi menggunakan data dengan durasi 10 detik, model BiLSTM memiliki waktu inferensi tercepat senilai 0,1467 detik.
Sementara itu, pada skenario inferensi menggunakan data dengan durasi 1 menit dan 10 menit, model LSTM-FCN memiliki waktu inferensi tercepat masing-masing senilai 0,8209 detik dan 10,1330 detik.
Tidak terdapat perbedaan yang signifikan pada penggunaan memori antara model LSTM-512, LSTM-256, BiLSTM, dan LSTM-FCN pada Raspberry Pi 4.
Penggunaan memori pada Raspberry Pi 4 pada inferensi menggunakan data dengan durasi 10 detik, 1 menit, dan 10 menit masing-masing sekitar 131 MB, 134 MB, dan 180 MB, dengan selisih kurang dari 1 MB.
Penggunaan memori tersebut masih dalam batas yang dapat diterima oleh Raspberry Pi 4 yang memiliki memori sebesar 8 GB.


\begin{table}[H]
\centering
\caption{Hasil evaluasi waktu inferensi dan penggunaan memori pada Intel NUC}
\label{tab:intelnuc}
\begin{tabular}{|c|c|c|c|}
\hline
\textbf{Model} & \textbf{Durasi ECG} & \textbf{Waktu Inferensi (s)} & \textbf{Penggunaan Memori (MB)} \\ \hline
\multirow{3}{*}{ LSTM-512 }       & 10s                 & 0,0360                      & 133,1211                   \\ 
              \cline{2-4}
               & 1m                  & 0,2760                      & 136,9086                   \\ 
              \cline{2-4}
               & 10m                 & 3,1574                      & \textbf{182,0367}                   \\ \hline
\multirow{3}{*}{ LSTM-256 }       & 10s                 & \textbf{0,0359}                      & \textbf{132,8539}                   \\
              \cline{2-4}
               & 1m                  & \textbf{0,2091}                      & \textbf{136,5195}                   \\ 
              \cline{2-4}
               & 10m                 & 2,2683                      & 182,2008                   \\ \hline
\multirow{3}{*}{ BiLSTM }         & 10s                 & 0,0371                      & 133,7625                   \\
              \cline{2-4}
               & 1m                  & 0,2360                      & 136,9438                   \\ 
              \cline{2-4}
               & 10m                 & 2,6397                      & 182,1641                   \\ \hline
\multirow{3}{*}{ LSTM-FCN }       & 10s                 & 0,0367                      & 133,6398                   \\
              \cline{2-4}
               & 1m                  & 0,2190                      & 137,0242                   \\
              \cline{2-4}
               & 10m                 & \textbf{1,9899}                      & 182,6266                   \\ \hline
\end{tabular}
\end{table}

Tabel \ref{tab:intelnuc} menunjukkan hasil evaluasi waktu inferensi dan penggunaan memori pada Intel NUC.
Pada skenario inferensi menggunakan data dengan durasi 10 detik dan 1 menit, model LSTM-256 memiliki waktu inferensi tercepat masing-masing senilai 0,0359 detik dan 0,2091 detik.
Sementara itu, pada skenario inferensi menggunakan data dengan durasi 10 menit, model LSTM-FCN memiliki waktu inferensi tercepat senilai 1,9899 detik.
Tidak terdapat perbedaan yang signifikan pada penggunaan memori antara model LSTM-512, LSTM-256, BiLSTM, dan LSTM-FCN pada Intel NUC.
Penggunaan memori pada Intel NUC pada inferensi menggunakan data dengan durasi 10 detik, 1 menit, dan 10 menit masing-masing sekitar 133 MB, 136 MB, dan 182 MB, dengan selisih kurang dari 1 MB.
Penggunaan memori tersebut masih dalam batas yang dapat diterima oleh Intel NUC yang memiliki memori sebesar 16 GB.

Berdasarkan hasil evaluasi waktu inferensi dan penggunaan memori pada Raspberry Pi 4 dan Intel NUC yang ditunjukkan pada Tabel \ref{tab:raspi4} dan Tabel \ref{tab:intelnuc}, dapat dilihat bahwa kecepatan waktu inferensi pada Intel NUC lebih cepat dibandingkan dengan Raspberry Pi 4.
Hal ini disebabkan oleh spesifikasi perangkat keras yang lebih tinggi pada Intel NUC dibandingkan dengan Raspberry Pi 4 terutama pada \textit{processor}.
% Terdapat perbedaan model yang memiliki waktu inferensi tercepat pada masing-masing perangkat.
% Pada Raspberry Pi 4, model BiLSTM memiliki waktu inferensi tercepat pada skenario inferensi menggunakan data dengan durasi 10 detik.
% Sementara itu, model LSTM-FCN memiliki waktu inferensi tercepat pada skenario inferensi menggunakan data dengan durasi 1 menit dan 10 menit.
% Pada Intel NUC, model LSTM-256 memiliki waktu inferensi tercepat pada skenario inferensi menggunakan data dengan durasi 10 detik dan 1 menit.
% Sementara itu, model LSTM-FCN memiliki waktu inferensi tercepat pada skenario inferensi menggunakan data dengan durasi 10 menit.
Sementara itu, penggunaan memori pada Raspberry Pi 4 dan Intel NUC memiliki selisih yang tidak signifikan.
Penggunaan memori pada Intel NUC sedikit lebih tinggi dibandingkan dengan Raspberry Pi 4 pada masing-masing skenario pengujian, dengan selisih sekitar 2 MB.




% --- tabel hasil prediksi pada pengujian ---
% model, pengujian ke-n, hasil prediksi (N, S, V, F, Q)
% ----------
% | Model | Pengujian ke | Hasil Prediksi |
% |       |        | N | S | V | F | Q |
% |-------|--------|----------------|
% 1 m								10m					
% lstm512	n	s	v	f	q			lstm512	n	s	v	f	q
% 1	59	0	0	0	0			1	978	0	0	0	0
% 2	62	0	0	0	0			2	969	0	0	0	0
% 3	62	0	0	0	0			3	969	0	0	0	0
% 4	60	0	0	0	0			4	969	0	0	0	0
% 5	58	0	0	0	0			5	969	0	0	0	0
% lstm256	n	s	v	f	q			lstm256	n	s	v	f	q
% 1	59	0	0	0	0			1	978	0	0	0	0
% 2	62	0	0	0	0			2	969	0	0	0	0
% 3	62	0	0	0	0			3	969	0	0	0	0
% 4	60	0	0	0	0			4	969	0	0	0	0
% 5	58	0	0	0	0			5	969	0	0	0	0
% bilstm	n	s	v	f	q			bilstm	n	s	v	f	q
% 1	59	0	0	0	0			1	978	0	0	0	0
% 2	62	0	0	0	0			2	969	0	0	0	0
% 3	62	0	0	0	0			3	969	0	0	0	0
% 4	60	0	0	0	0			4	969	0	0	0	0
% 5	58	0	0	0	0			5	969	0	0	0	0
% lstmfcn	n	s	v	f	q			lstmfcn	n	s	v	f	q
% 1	59	0	0	0	0			1	978	0	0	0	0
% 2	62	0	0	0	0			2	969	0	0	0	0
% 3	62	0	0	0	0			3	969	0	0	0	0
% 4	60	0	0	0	0			4	969	0	0	0	0
% 5	58	0	0	0	0			5	969	0	0	0	0

% -- tabel jumlah detak jantung yang diprediksi pada pengujian --
% durasi ecg, jumlah detak jantung yang diprediksi
% ----------

\begin{table}[H]
\centering
\caption{Jumlah detak jantung yang diprediksi pada pengujian}
\label{tab:jumlah-detak-jantung}
\begin{tabularx}{0.8\textwidth}{
  |>{\centering\arraybackslash}X
  |c
|}
\hline
\textbf{Durasi ECG} & \textbf{Jumlah Detak Jantung yang Diprediksi} \\ \hline
10s                 & 0                                              \\
\hline
1m                  & 301                                            \\
\hline
10m                 & 4854                                           \\ \hline
\end{tabularx}
\end{table}


% Tabel \ref{tab:jumlah-detak-jantung} menunjukkan jumlah detak jantung yang diprediksi selama pengujian pada data ECG dengan durasi 10 detik, 1 menit, dan 10 menit.
Tabel \ref{tab:jumlah-detak-jantung} menunjukkan jumlah total detak jantung yang diprediksi selama lima kali pengujian pada masing-masing skenario pengujian.
% Jumlah detak jantung yang diprediksi selama pengujian pada data ECG dengan durasi 10 detik, 1 menit, dan 10 menit ditunjukkan pada Tabel \ref{tab:jumlah-detak-jantung}.
Pada skenario inferensi menggunakan data dengan durasi 10 detik, tidak ada detak jantung yang diprediksi.
Hal ini disebabkan oleh jumlah detak jantung yang terlalu sedikit pada data dengan durasi 10 detik.
Jumlah detak jantung yang terlalu sedikit menyebabkan tidak dapat dilakukan ekstraksi fitur karena pada tahap ekstraksi fitur membutuhkan minimal 43 RR-interval atau 44 detak jantung.
Sementara itu, pada skenario inferensi menggunakan data dengan durasi 1 menit dan 10 menit, jumlah total detak jantung yang diprediksi sebanyak 301 dan 4854 detak jantung.



% --- waktu inferensi per beat ---
    % for i in range(len(rr_intervals)):
    %     if i < 42 or i == len(rr_intervals) - 1:
    %         continue
% 42 beats will produce 41 rr intervals
% so it will need minimum of
Dari hasil pengujian pada skenario inferensi menggunakan data dengan durasi 1 menit dan 10 menit, dapat dihitung waktu inferensi per detak jantung pada masing-masing model seperti yang ditunjukkan pada Tabel \ref{tab:waktu-inferensi-per-beat-raspi4} dan Tabel \ref{tab:waktu-inferensi-per-beat-intelnuc}.
Apabila dihitung waktu inferensi per detak jantung, model LSTM-FCN memiliki waktu inferensi per detak jantung tercepat, baik pada Raspberry Pi 4 maupun Intel NUC.
Pada Raspberry Pi 4, model LSTM-FCN memiliki waktu inferensi per detak jantung sebesar 12,0372 ms, sedangkan pada Intel NUC, model LSTM-FCN memiliki waktu inferensi per detak jantung sebesar 2,8435 ms.
% posible for real-time application
Waktu inferensi per detak jantung pada Raspberry Pi 4 dan Intel NUC yang relatif cepat memungkinkan model dapat digunakan untuk aplikasi \textit{real-time}.

% normal resting heart rate is 50-90 bpm and can be as low as 30 bpm in athletes
% so the minimum rr interval is 30/60 = 0.5 s
% that means, for real-time application, the maximum time to process 1 beat is 0.5 s

%rpi
%lstm512 lstm256 bilstm lstmfcn
% 15.8633633491373						13.6235298022468						14.3540212386035						12.0371508949366

%nuc
%lstm512 lstm256 bilstm lstmfcn
% 3.91885739357172						2.90504177306471						3.31934236690784						2.84347337445122

\begin{table}[H]
\centering
\caption{Waktu inferensi per detak jantung pada Raspberry Pi 4}
\label{tab:waktu-inferensi-per-beat-raspi4}
\begin{tabularx}{0.8\textwidth}{
  |>{\centering\arraybackslash}X
  |c
|}
\hline
\textbf{Model} & \textbf{Waktu Inferensi per Detak Jantung (ms)} \\ \hline
LSTM-512       & 15,8634                   \\
\hline
LSTM-256       & 13,6235                   \\
\hline
BiLSTM         & 14,3540                   \\
\hline
LSTM-FCN       & \textbf{12,0372}                   \\ \hline
\end{tabularx}
\end{table}

\begin{table}[H]
\centering
\caption{Waktu inferensi per detak jantung pada Intel NUC}
\label{tab:waktu-inferensi-per-beat-intelnuc}
\begin{tabularx}{0.8\textwidth}{
  |>{\centering\arraybackslash}X
  |c
|}
\hline
\textbf{Model} & \textbf{Waktu Inferensi per Detak Jantung (ms)} \\ \hline
LSTM-512       & 3,9189                   \\
\hline
LSTM-256       & 2,9050                   \\
\hline
BiLSTM         & 3,3193                   \\
\hline
LSTM-FCN       & \textbf{2,8435}                   \\ \hline
\end{tabularx}
\end{table}

% --- komparasi dengan penelitian terkait ---

% \begin{table}[H]
% \centering
% \caption{Komparasi dengan penelitian terdahulu}
% \label{tab:komparasi}
% \begin{tabular}{ccccc}
% \hline
% \textbf{Penelitian} & \textbf{Perangkat} & \textbf{Akurasi} & \textbf{Waktu Inferensi} & \textbf{Penggunaan Memori} \\ \hline
%
% \cite{saadatnejadLSTMBasedECGClassification2020} & Moto 360 (ARM Cortex-A7) & 0.9741 & 31.2ms & - \\
%
% % Penelitian ini & RR
% \end{tabular}
% \end{table}

% use tabularx instead

\begin{table}[H]
\centering
\caption{Komparasi dengan penelitian terdahulu}
\label{tab:komparasi}
% \begin{tabularx}{\textwidth}{>{\raggedright\arraybackslash}XXccc}
\begin{tabularx}{\textwidth}{
  |>{\raggedright\arraybackslash}X
  |>{\centring\arraybackslash}X
  |>{\raggedright\arraybackslash}X
  |c
  |c
  % |>{\raggedright\arraybackslash}X
  % |>{\raggedright\arraybackslash}X
|}
\hline
% \textbf{Penelitian} & \textbf{Perangkat} & \textbf{Akurasi} & \makecell{\textbf{Waktu} \\ \textbf{Inferensi}} & \makecell{\textbf{Penggunaan} \\ \textbf{Memori}} \\ \hline
\textbf{Penelitian} & \textbf{Model} & \textbf{Perangkat} & \textbf{Akurasi} & \makecell{\textbf{Waktu} \\ \textbf{Inferensi (ms)}} \\ \hline

\cite{saadatnejadLSTMBasedECGClassification2020} & LSTM & Moto 360 (ARM Cortex-A7) & 0,9741 & 31,2\\
\hline

\cite{9878113} & Conv1D + Conv2D & Raspberry Pi & 0,991 & 9\\
\hline

% 21.1326 s / 3267
\cite{sururiComparisonSeveralWavelet2023} & CNN & Intel Core i5 & 0,9908 & 6,47\\
\hline

\cite{FALASCHETTI20223479} & LSTM & STM32L4 (ARM Cortex-M4) & 0,9019 & 665,86\\
\hline

\cite{liEnablingOndeviceClassification2021} & 1-D CNN & Raspberry Pi Zero & 0,9835 & 7,08\\
\hline

\cite{heLiteNetLightweightNeural2018} & LiteNet & Intel Core i3-2370M & 0,9787 & \tilde 25\\
\hline

\cite{abayaratneRealTimeCardiacArrhythmia2019} & RNN-LSTM & - & 0,947 & 6,88\\
\hline

\cite{mhamdiArtificialIntelligenceCardiac2022} & MobileNetV2 & Raspberry Pi 4 B & 0,94 & 160\\
\hline

\textbf{Penelitian ini} & LSTM-256 & Raspberry Pi 4 B (ARM Cortex-A72) & 0,9679 & 15,86\\
\hline
\textbf{Penelitian ini} & LSTM-256 & Intel NUC (Intel Core i3-1115G4) & 0,9679 & 3,92\\ 

\hline
\end{tabularx}
\end{table}

Tabel \ref{tab:komparasi} menunjukkan komparasi hasil penelitian ini dengan beberapa penelitian terdahulu.
Komparasi dilakukan berdasarkan akurasi, waktu inferensi, dan perangkat yang digunakan pada penelitian terdahulu.
Model pada penelitian ini yang digunakan untuk komparasi adalah model LSTM-256.
Model LSTM-256 dipilih karena memiliki akurasi tertinggi nomor dua setelah model LSTM-512 dengan selisih akurasi yang tidak signifikan serta memiliki waktu inferensi tercepat nomor dua setelah model LSTM-FCN baik pada Raspberry Pi 4 maupun Intel NUC.
Komparasi tersebut menunjukkan bahwa model LSTM-256 yang digunakan pada penelitian ini memiliki akurasi serta waktu inferensi yang kompetitif dibandingkan dengan penelitian terdahulu.
Model LSTM-256 menduduki peringkat enam dalam hal akurasi.
Sementara itu, model LSTM-256 pada Intel NUC menduduki peringkat pertama dalam hal waktu inferensi.

